\pagestyle{plain}
\begin{center}
  \textsc{Zusammenfassung}
\end{center}
%
\begin{otherlanguage}{ngerman}
Die Netzwerkanalyse ist eine Sammlung von Techniken, die darauf abzielen,
nicht-triviale Erkenntnisse aus vernetzten Daten durch die Untersuchung von
Beziehungsmustern zwischen den Entitäten eines Netzwerks zu gewinnen.
Zu den Erkenntnissen, die aus einem Netzwerk
gewonnen werden können, gehört die Bestimmung der Wichtigkeit von Entitäten
an Hand von definierten Kriterien -- in sozialen Netzwerken gibt es
beispielsweise in der Regel einige Teilnehmer, die einflussreicher sind als
andere oder deren Entfernung aus dem Netzwerk eine erhebliche Veränderung des
Kommunikationsflusses bedeuten würde. Eine andere Möglichkeit besteht darin,
für jeden Teilnehmer eines Netzwerks den am besten geeigneten Partner zu
finden, wenn man die paarweisen Präferenzen (oder die Kompatibilität) der
Teilnehmer kennt, die miteinander verbunden werden sollen -- auch bekannt als
das Maximum Weighted Matching-Problem (MWM). Beispiele hierfür sind Plattformen
für Matchmaking oder die Paarung von Spielern in einem Schachturnier.

Die Wichtigkeit ist hierbei stark an die jeweilige Anwendung gebunden.
Daher wurden in den letzten Jahren mehrere \emph{Zentralitätsmaße} eingeführt.
Diese Maße
stammen hierbei aus Jahrzehnten, in denen die Rechenleistung sehr begrenzt war
und die Netzwerke im Vergleich zu heute viel kleiner waren -- Skalierbarkeit auf
große Datenmengen wurde daher nicht berücksichtigt.
Heutzutage sind jedoch Netzwerke mit Millionen von Kanten
allgegenwärtig und eine vollständige exakte Berechnung vieler traditioneller
Zentralitätsmaße -- die immer noch weit verbreitet sind -- ist zu zeitaufwendig.
Dieses Problem wird noch verstärkt, wenn das Ziel darin besteht, die
\emph{Gruppe} der $k$-Knoten mit der höchsten gemeinsamen Zentralität zu
finden; dieses Problem hat nützliche Anwendungen bei der Optimierung von
Standorten von Lagern und der Einflussmaximierung.
Skalierbare Algorithmen
zur Identifizierung hochzentraler (Gruppen von) Knoten in großen Graphen sind von
zentraler Bedeutung für eine umfassende Netzwerkanalyse.
%
Die heutigen Netzwerke sind nicht nur groß, sondern
verändern sich zusätzlich im zeitlichen Verlauf. Daraus ergibt
sich die Herausforderung, die Erkenntnisse, die aus dem Netzwerk
gewonnen wurden, nach einer Änderung effizient zu
aktualisieren. Effiziente \emph{dynamische} Algorithmen
sind daher ein weiterer wesentlicher Bestandteil moderner
Analyse-Pipelines.

Hauptziel dieser Arbeit ist es, skalierbare algorithmische
Lösungen für die zwei bereits genannten Herausforderungen zu liefern: die
Identifizierung wichtiger Knotenpunkte in einem Netzwerk und deren
effiziente Aktualisierung in sich verändernden Netzwerken.
Die meisten unserer Algorithmen benötigen Sekunden bis einige
Minuten, um diese Aufgaben in realen Netzwerken mit bis zu Hunderten Millionen
von Kanten zu Lösen, was eine deutliche Verbesserung gegenüber dem Stand der Technik
darstellt.

Die Berechnung von MWMs in großen Netzwerke ist rechenintensiv.
Aus diesem Grund gibt es in der Literatur zahlreiche
schnelle inexakte Algorithmen für MWM, sowie dynamische Algorithmen zur
Aufrechterhaltung eines (approximativen) MWMs in dynamischen Graphen. Es wurden
jedoch nur wenige Anstrengungen unternommen, um diese Algorithmen in der Praxis
zu implementieren, insbesondere im dynamischen Fall, so dass ihre tatsächliche
Leistung unbekannt ist. Daher besteht ein weiteres Ziel dieser Arbeit darin,
die Lücke zwischen Theorie und Praxis im Zusammenhang mit dynamischen MWM zu
schließen. Insbesondere entwickeln wir einen Algorithmus, der ein
approximatives MWM nach mehreren Kantenaktualisierungen von Graphen mit
Milliarden von Kanten in nur einem Bruchteil einer Sekunde aktualisiert.
\end{otherlanguage}
