\apxch{ch:group-harm-clos-max}

\section{Approximation for Group-Closeness in the Sense of Li \etal}
\label{apx:gh-gc:li-etal-apx}
%
As explained in Ref.~\cite[Appendix C]{DBLP:conf/alenex/AngrimanBDGGM21},
the approach of Li \etal~\cite{DBLP:conf/www/0002PSYZ19} works for
minimizing a general supermodular monotone non-increasing function
$f(\cdot)$ with respect to a cardinality constraint.
They let $x^*_1 \in \argmax_x\set{f(\emptyset) - f(\set{x})}$ and use
the greedy algorithm on the set function
%
\[
    g(S) := f(\set{x^*_1}) - f(\set{x^*_1} \cup S),
\]
%
which is a monotone non-decreasing submodular set function with
$g(\emptyset) = 0$. Thus, the greedy algorithm maximizes the function
with respect to a cardinality constraint within an approximation factor
of $1 - 1/e$~\cite{DBLP:journals/mp/NemhauserWF78}.
However, there are two caveats. First, the greedy algorithm uses a budget
of $k - 1$ instead of $k$ (budget of one is spent on identifying $x^*_1$)
and thus Li \etal obtain an approximation factor of $1 - k/((k - 1)e)$.
Second, and most importantly, observe that the approximation factor is
obtained on the function $g(S)$ and not $f(S)$, \ie they
get a set $S$ of size $k - 1$ such that
%
\[
f(\set{x^*_1}) - f(\set{x^*_1} \cup S) \ge
\roundb{1 - \frac{k}{(k - 1)e}}\cdot(f(\set{x^*_1}) - f(\set{x^*_1} \cup S^*)),
\]
%
where $S^*$ is the optimal set of size $k - 1$ to be added to $\set{x^*_1}$
with the goal of minimizing $f(\cdot)$.
We remark that this set is not necessarily related to the set that minimizes
$f(\cdot)$ with respect to the cardinality constraint. Clearly, this approach
can be applied for the submodular farness function $\gfarn(\cdot)$ in place
of $f(\cdot)$. It can not, however, provide an approximation algorithm
for $\gfarn(\cdot)$ in the usual sense -- and furthermore it would not be
easily extendable to $\gclos(\cdot)$.
\vfill\pagebreak

\section{Instances Statistics}
\label{sec:gh-gc:inst-stats}
\setlength{\tabcolsep}{2pt}

\begin{table}[H]
\footnotesize\centering
\captionabove{Small networks used for group-harmonic experiments with ILP solver.}
\label{tab:gh-gc-apx:small-inst-gh}
\begin{subtable}[t]{.45\textwidth}
\centering
\caption{Complex networks}
\label{tab:gh-gc-apx:small-inst-gh-cplx}
\begin{tabular}{lcrr}
\toprule
Graph & Type & $n$ & $m$\\
\midrule
convote & \texttt{U} & \numprint{219} & \numprint{586}\\
dimacs10-football & \texttt{U} & \numprint{115} & \numprint{613}\\
wiki\_talk\_ht & \texttt{U} & \numprint{537} & \numprint{787}\\
moreno\_innovation & \texttt{U} & \numprint{241} & \numprint{1098}\\
dimacs10-celegans\_metabolic & \texttt{U} & \numprint{453} & \numprint{2025}\\
arenas-meta & \texttt{U} & \numprint{453} & \numprint{2025}\\
foodweb-baywet & \texttt{U} & \numprint{128} & \numprint{2106}\\
contact & \texttt{U} & \numprint{275} & \numprint{2124}\\
foodweb-baydry & \texttt{U} & \numprint{128} & \numprint{2137}\\
moreno\_oz & \texttt{U} & \numprint{217} & \numprint{2672}\\
arenas-jazz & \texttt{U} & \numprint{198} & \numprint{2742}\\
sociopatterns-infectious & \texttt{U} & \numprint{411} & \numprint{2765}\\
dimacs10-celegansneural & \texttt{U} & \numprint{297} & \numprint{4296}\\
radoslaw\_email & \texttt{U} & \numprint{168} & \numprint{5783}\\
\midrule
convote & \texttt{D} & \numprint{219} & \numprint{586}\\
wiki\_talk\_ht & \texttt{D} & \numprint{537} & \numprint{787}\\
moreno\_innovation & \texttt{D} & \numprint{241} & \numprint{1098}\\
foodweb-baywet & \texttt{D} & \numprint{128} & \numprint{2106}\\
foodweb-baydry & \texttt{D} & \numprint{128} & \numprint{2137}\\
moreno\_oz & \texttt{D} & \numprint{217} & \numprint{2672}\\
dimacs10-celegansneural & \texttt{D} & \numprint{297} & \numprint{4296}\\
radoslaw\_email & \texttt{D} & \numprint{168} & \numprint{5783}\\
\bottomrule
\end{tabular}

\end{subtable}\hfill
\begin{subtable}[t]{.45\textwidth}
\centering
\caption{High-diameter networks}
\label{tab:gh-gc-apx:small-inst-gh-high-diam}
\begin{tabular}{lcrr}
\toprule
Graph & Type & $n$ & $m$\\
\midrule
dbpedia-similar & \texttt{UU} & \numprint{430} & \numprint{564}\\
niue & \texttt{UU} & \numprint{461} & \numprint{1055}\\
tuvalu & \texttt{UU} & \numprint{436} & \numprint{1082}\\
librec-filmtrust-trust & \texttt{UU} & \numprint{874} & \numprint{1853}\\
\midrule
niue & \texttt{UW} & \numprint{461} & \numprint{1055}\\
tuvalu & \texttt{UW} & \numprint{436} & \numprint{1082}\\
\midrule
niue & \texttt{DU} & \numprint{461} & \numprint{1055}\\
tuvalu & \texttt{DU} & \numprint{436} & \numprint{1082}\\
librec-filmtrust-trust & \texttt{DU} & \numprint{874} & \numprint{1853}\\
\midrule
niue & \texttt{DW} & \numprint{461} & \numprint{1055}\\
tuvalu & \texttt{DW} & \numprint{436} & \numprint{1082}\\
\bottomrule
\end{tabular}

\end{subtable}
\end{table}

\begin{table}[tb]
\footnotesize\centering
\captionabove{Small networks used for group-closeness experiments with ILP solver.}
\label{tab:gh-gc-apx:small-inst-gc}
\begin{subtable}[t]{.45\textwidth}
\centering
\caption{Complex networks}
\begin{tabular}{lcrr}
\toprule
Graph & Type & $n$ & $m$\\
\midrule
dimacs10-celegans\_metabolic & \texttt{U} & \numprint{453} & \numprint{2025}\\
arenas-meta & \texttt{U} & \numprint{453} & \numprint{2025}\\
contact & \texttt{U} & \numprint{274} & \numprint{2124}\\
arenas-jazz & \texttt{U} & \numprint{198} & \numprint{2742}\\
sociopatterns-infectious & \texttt{U} & \numprint{410} & \numprint{2765}\\
dnc-corecipient & \texttt{U} & \numprint{849} & \numprint{10384}\\
\midrule
moreno\_oz & \texttt{D} & \numprint{214} & \numprint{2658}\\
wiki\_talk\_lv & \texttt{D} & \numprint{510} & \numprint{2783}\\
wiki\_talk\_eu & \texttt{D} & \numprint{617} & \numprint{2811}\\
dnc-temporalGraph & \texttt{D} & \numprint{520} & \numprint{3518}\\
dimacs10-celegansneural & \texttt{D} & \numprint{297} & \numprint{4296}\\
wiki\_talk\_bn & \texttt{D} & \numprint{700} & \numprint{4316}\\
wiki\_talk\_eo & \texttt{D} & \numprint{822} & \numprint{6076}\\
wiki\_talk\_gl & \texttt{D} & \numprint{1009} & \numprint{7435}\\
\bottomrule
\end{tabular}

\end{subtable}\hfill
\begin{subtable}[t]{.45\textwidth}
\centering
\caption{High-diameter networks}
\begin{tabular}{lcrr}
\toprule
Graph & Type & $n$ & $m$\\
\midrule
tuvalu & \texttt{UU} & \numprint{152} & \numprint{187}\\
niue & \texttt{UU} & \numprint{461} & \numprint{529}\\
nauru & \texttt{UU} & \numprint{618} & \numprint{729}\\
dimacs10-netscience & \texttt{UU} & \numprint{379} & \numprint{914}\\
asoiaf & \texttt{UU} & \numprint{796} & \numprint{2823}\\
\midrule
tuvalu & \texttt{UW} & \numprint{152} & \numprint{187}\\
niue & \texttt{UW} & \numprint{461} & \numprint{529}\\
nauru & \texttt{UW} & \numprint{618} & \numprint{729}\\
\midrule
tuvalu & \texttt{DU} & \numprint{152} & \numprint{374}\\
niue & \texttt{DU} & \numprint{461} & \numprint{1055}\\
librec-filmtrust-trust & \texttt{DU} & \numprint{267} & \numprint{1099}\\
nauru & \texttt{DU} & \numprint{618} & \numprint{1427}\\
\midrule
tuvalu & \texttt{DW} & \numprint{152} & \numprint{374}\\
niue & \texttt{DW} & \numprint{461} & \numprint{1055}\\
nauru & \texttt{DW} & \numprint{618} & \numprint{1427}\\
\bottomrule
\end{tabular}

\end{subtable}
\end{table}

\begin{table}[tb]
\footnotesize
\captionabove{Large networks used for group-harmonic experiments.}
\label{tab:gh-gc-apx:large-inst-gh}
\begin{subtable}[t]{.45\textwidth}
\caption{Complex networks}
\label{tab:gh-gc-apx:large-inst-gh-cplx}
\begin{tabular}{lcrr}
\toprule
Graph & Type & $n$ & $m$\\
\midrule
petster-hamster-household & \texttt{U} & \numprint{874} & \numprint{4003}\\
petster-hamster-friend & \texttt{U} & \numprint{1788} & \numprint{12476}\\
petster-hamster & \texttt{U} & \numprint{2000} & \numprint{16098}\\
loc-brightkite\_edges & \texttt{U} & \numprint{58228} & \numprint{214078}\\
douban & \texttt{U} & \numprint{154908} & \numprint{327162}\\
petster-cat-household & \texttt{U} & \numprint{105138} & \numprint{494858}\\
loc-gowalla\_edges & \texttt{U} & \numprint{196591} & \numprint{950327}\\
wikipedia\_link\_fy & \texttt{U} & \numprint{65562} & \numprint{1071668}\\
wikipedia\_link\_ckb & \texttt{U} & \numprint{60722} & \numprint{1176289}\\
petster-dog-household & \texttt{U} & \numprint{260390} & \numprint{2148179}\\
livemocha & \texttt{U} & \numprint{104103} & \numprint{2193083}\\
flickrEdges & \texttt{U} & \numprint{105938} & \numprint{2316948}\\
petster-friendships-cat & \texttt{U} & \numprint{149700} & \numprint{5448197}\\
\midrule
wikipedia\_link\_mi & \texttt{D} & \numprint{7996} & \numprint{116457}\\
foldoc & \texttt{D} & \numprint{13356} & \numprint{120238}\\
wikipedia\_link\_so & \texttt{D} & \numprint{7439} & \numprint{125046}\\
wikipedia\_link\_lo & \texttt{D} & \numprint{3811} & \numprint{132837}\\
wikipedia\_link\_co & \texttt{D} & \numprint{8252} & \numprint{177420}\\
\bottomrule
\end{tabular}

\end{subtable}\hfill
\begin{subtable}[t]{.45\textwidth}
\caption{High-diameter networks}
\label{tab:gh-gc-apx:large-inst-gh-high-diam}
\begin{tabular}{lcrr}
\toprule
Graph & Type & $n$ & $m$\\
\midrule
marshall-islands & \texttt{UU} & \numprint{1080} & \numprint{2557}\\
micronesia & \texttt{UU} & \numprint{1703} & \numprint{3600}\\
kiribati & \texttt{UU} & \numprint{1867} & \numprint{4412}\\
opsahl-powergrid & \texttt{UU} & \numprint{4941} & \numprint{6594}\\
samoa & \texttt{UU} & \numprint{6926} & \numprint{15217}\\
comores & \texttt{UU} & \numprint{7250} & \numprint{17554}\\
\midrule
marshall-islands & \texttt{UW} & \numprint{1080} & \numprint{2557}\\
micronesia & \texttt{UW} & \numprint{1703} & \numprint{3600}\\
kiribati & \texttt{UW} & \numprint{1867} & \numprint{4412}\\
DC & \texttt{UW} & \numprint{9522} & \numprint{14807}\\
samoa & \texttt{UW} & \numprint{6926} & \numprint{15217}\\
comores & \texttt{UW} & \numprint{7250} & \numprint{17554}\\
\midrule
marshall-islands & \texttt{DU} & \numprint{1080} & \numprint{2557}\\
micronesia & \texttt{DU} & \numprint{1703} & \numprint{3600}\\
kiribati & \texttt{DU} & \numprint{1867} & \numprint{4412}\\
samoa & \texttt{DU} & \numprint{6926} & \numprint{15217}\\
comores & \texttt{DU} & \numprint{7250} & \numprint{17554}\\
opsahl-openflights & \texttt{DU} & \numprint{2939} & \numprint{30501}\\
tntp-ChicagoRegional & \texttt{DU} & \numprint{12982} & \numprint{39018}\\
\midrule
marshall-islands & \texttt{DW} & \numprint{1080} & \numprint{2557}\\
micronesia & \texttt{DW} & \numprint{1703} & \numprint{3600}\\
kiribati & \texttt{DW} & \numprint{1867} & \numprint{4412}\\
samoa & \texttt{DW} & \numprint{6926} & \numprint{15217}\\
comores & \texttt{DW} & \numprint{7250} & \numprint{17554}\\
\bottomrule
\end{tabular}

\end{subtable}
\end{table}

\begin{table}[tb]
\footnotesize
\captionabove{Large (strongly) connected components of the networks in \Cref{tab:gh-gc-apx:large-inst-gh}
used for group-closeness experiments.}
\label{tab:gh-gc-apx:large-inst-gc}
\begin{subtable}[t]{.45\textwidth}
\caption{Complex networks}
\label{tab:gh-gc-apx:large-inst-gc-cplx}
\begin{tabular}{lcrr}
\toprule
Graph & Type & $n$ & $m$\\
\midrule
loc-brightkite\_edges & \texttt{U} & \numprint{56739} & \numprint{212945}\\
douban & \texttt{U} & \numprint{154908} & \numprint{327162}\\
petster-cat-household & \texttt{U} & \numprint{68315} & \numprint{494562}\\
wikipedia\_link\_ckb & \texttt{U} & \numprint{60257} & \numprint{801794}\\
wikipedia\_link\_fy & \texttt{U} & \numprint{65512} & \numprint{921533}\\
livemocha & \texttt{U} & \numprint{104103} & \numprint{2193083}\\
\midrule
wikipedia\_link\_mi & \texttt{D} & \numprint{3696} & \numprint{99237}\\
wikipedia\_link\_lo & \texttt{D} & \numprint{1622} & \numprint{109577}\\
wikipedia\_link\_so & \texttt{D} & \numprint{5149} & \numprint{114922}\\
foldoc & \texttt{D} & \numprint{13274} & \numprint{119485}\\
wikipedia\_link\_co & \texttt{D} & \numprint{5150} & \numprint{160474}\\
web-NotreDame & \texttt{D} & \numprint{53968} & \numprint{296228}\\
slashdot-zoo & \texttt{D} & \numprint{26997} & \numprint{333425}\\
soc-Epinions1 & \texttt{D} & \numprint{32223} & \numprint{443506}\\
wikipedia\_link\_jv & \texttt{D} & \numprint{39248} & \numprint{1059059}\\
\bottomrule
\end{tabular}

\end{subtable}\hfill
\begin{subtable}[t]{.45\textwidth}
\caption{High-diameter networks}
\label{tab:gh-gc-apx:large-inst-gc-high-diam}
\begin{tabular}{lcrr}
\toprule
Graph & Type & $n$ & $m$\\
\midrule
seychelles & \texttt{UU} & \numprint{3907} & \numprint{4322}\\
comores & \texttt{UU} & \numprint{3789} & \numprint{4630}\\
andorra & \texttt{UU} & \numprint{4219} & \numprint{4933}\\
opsahl-powergrid & \texttt{UU} & \numprint{4941} & \numprint{6594}\\
liechtenstein & \texttt{UU} & \numprint{6215} & \numprint{7002}\\
faroe-islands & \texttt{UU} & \numprint{12129} & \numprint{13165}\\
\midrule
seychelles & \texttt{UW} & \numprint{3907} & \numprint{4322}\\
comores & \texttt{UW} & \numprint{3789} & \numprint{4630}\\
andorra & \texttt{UW} & \numprint{4219} & \numprint{4933}\\
liechtenstein & \texttt{UW} & \numprint{6215} & \numprint{7002}\\
faroe-islands & \texttt{UW} & \numprint{12129} & \numprint{13165}\\
DC & \texttt{UW} & \numprint{9522} & \numprint{14807}\\
\midrule
seychelles & \texttt{DU} & \numprint{3907} & \numprint{8225}\\
andorra & \texttt{DU} & \numprint{4160} & \numprint{8288}\\
comores & \texttt{DU} & \numprint{3789} & \numprint{8952}\\
liechtenstein & \texttt{DU} & \numprint{6205} & \numprint{13591}\\
faroe-islands & \texttt{DU} & \numprint{12077} & \numprint{25679}\\
opsahl-openflights & \texttt{DU} & \numprint{2868} & \numprint{30404}\\
tntp-ChicagoRegional & \texttt{DU} & \numprint{12978} & \numprint{39017}\\
\midrule
seychelles & \texttt{DW} & \numprint{3907} & \numprint{8225}\\
andorra & \texttt{DW} & \numprint{4160} & \numprint{8288}\\
comores & \texttt{DW} & \numprint{3789} & \numprint{8952}\\
liechtenstein & \texttt{DW} & \numprint{6205} & \numprint{13591}\\
faroe-islands & \texttt{DW} & \numprint{12077} & \numprint{25679}\\
\bottomrule
\end{tabular}

\end{subtable}
\end{table}

\FloatBarrier

\section{Running Times -- Group-Harmonic Maximization}
\label{sec:gh-gc:running-times-gh}

\begin{table}[H]
\scriptsize
\captionabove{Running times (s) of \greedyh and \greedylsh
on the complex networks of \Cref{tab:gh-gc-apx:large-inst-gh-cplx}.}
\label{tab:gh-gc-apx:time-gh-cplx}

\begin{subtable}[t]{.6\textwidth}
\centering
\caption{Undirected unweighted}
\begin{tabular}{lrrrrrr}
\toprule
Graph & \multicolumn{3}{c}{Greedy-H} & \multicolumn{3}{c}{Greedy-LS-H}\\
\hfill $k$ & $5$ & $10$ & $50$ & $5$ & $10$ & $50$\\
\midrule
petster-hamster-household & \textless\numprint{0.1} & \textless\numprint{0.1} & \textless\numprint{0.1} & \textless\numprint{0.1} & \textless\numprint{0.1} & \textless\numprint{0.1}\\
petster-hamster-friend & \textless\numprint{0.1} & \textless\numprint{0.1} & \textless\numprint{0.1} & \textless\numprint{0.1} & \textless\numprint{0.1} & \numprint{0.1}\\
petster-hamster & \textless\numprint{0.1} & \textless\numprint{0.1} & \textless\numprint{0.1} & \textless\numprint{0.1} & \textless\numprint{0.1} & \textless\numprint{0.1}\\
loc-brightkite\_edges & \numprint{1.1} & \numprint{1.0} & \numprint{1.1} & \numprint{4.3} & \numprint{6.6} & \numprint{25.8}\\
douban & \numprint{8.1} & \numprint{8.1} & \numprint{8.4} & \numprint{40.3} & \numprint{86.3} & \numprint{303.0}\\
petster-cat-household & \numprint{0.1} & \numprint{0.2} & \numprint{0.3} & \numprint{19.5} & \numprint{23.8} & \numprint{106.1}\\
loc-gowalla\_edges & \numprint{8.9} & \numprint{8.4} & \numprint{8.7} & \numprint{59.8} & \numprint{97.3} & \numprint{1064.5}\\
wikipedia\_link\_fy & \numprint{3.8} & \numprint{3.8} & \numprint{4.0} & \numprint{13.3} & \numprint{15.7} & \numprint{137.9}\\
wikipedia\_link\_ckb & \numprint{7.3} & \numprint{7.3} & \numprint{7.4} & \numprint{12.9} & \numprint{14.6} & \numprint{80.2}\\
petster-dog-household & \numprint{10.3} & \numprint{10.4} & \numprint{10.7} & \numprint{131.9} & \numprint{212.3} & \numprint{843.8}\\
livemocha & \numprint{11.2} & \numprint{11.4} & \numprint{11.8} & \numprint{52.5} & \numprint{64.6} & \numprint{277.9}\\
flickrEdges & \numprint{44.2} & \numprint{45.4} & \numprint{46.4} & \numprint{119.5} & \numprint{128.4} & \numprint{217.6}\\
petster-friendships-cat & \numprint{2.7} & \numprint{2.8} & \numprint{2.9} & \numprint{35.6} & \numprint{55.1} & \numprint{266.7}\\
\bottomrule
\end{tabular}

\end{subtable}\hfill
\begin{subtable}[t]{.4\textwidth}
\centering
\caption{Directed unweighted}
\begin{tabular}{lrrrrrr}
\toprule
Graph & \multicolumn{3}{c}{Greedy-H} & \multicolumn{3}{c}{Greedy-LS-H}\\
\hfill $k$ & $5$ & $10$ & $50$ & $5$ & $10$ & $50$\\
\midrule
wikipedia\_link\_mi & \numprint{0.3} & \numprint{0.3} & \numprint{0.3} & \numprint{0.6} & \numprint{1.0} & \numprint{3.8}\\
foldoc & \numprint{0.6} & \numprint{0.6} & \numprint{0.6} & \numprint{1.6} & \numprint{1.7} & \numprint{14.7}\\
wikipedia\_link\_so & \numprint{0.1} & \numprint{0.1} & \numprint{0.1} & \numprint{0.3} & \numprint{0.4} & \numprint{2.2}\\
wikipedia\_link\_lo & \numprint{0.2} & \numprint{0.2} & \numprint{0.2} & \numprint{0.2} & \numprint{0.2} & \numprint{0.7}\\
wikipedia\_link\_co & \numprint{0.2} & \numprint{0.3} & \numprint{0.3} & \numprint{0.5} & \numprint{0.5} & \numprint{2.6}\\
\bottomrule
\end{tabular}

\end{subtable}
\end{table}

\begin{table}[H]
\scriptsize
\captionabove{Running times (s) of \greedyh and \greedylsh
on the high-diameter networks of \Cref{tab:gh-gc-apx:large-inst-gh-high-diam}.}
\label{tab:gh-gc-apx:time-gh-high-diam}

\begin{subtable}[t]{.5\textwidth}
\centering
\caption{Undirected unweighted}
\begin{tabular}{lrrrrrr}
\toprule
Graph & \multicolumn{3}{c}{Greedy-H} & \multicolumn{3}{c}{Greedy-LS-H}\\
\hfill $k$ & $5$ & $10$ & $50$ & $5$ & $10$ & $50$\\
\midrule
marshall-islands & \textless\numprint{0.1} & \textless\numprint{0.1} & \textless\numprint{0.1} & \textless\numprint{0.1} & \textless\numprint{0.1} & \textless\numprint{0.1}\\
micronesia & \textless\numprint{0.1} & \textless\numprint{0.1} & \textless\numprint{0.1} & \textless\numprint{0.1} & \textless\numprint{0.1} & \numprint{0.2}\\
kiribati & \textless\numprint{0.1} & \textless\numprint{0.1} & \textless\numprint{0.1} & \textless\numprint{0.1} & \textless\numprint{0.1} & \numprint{0.3}\\
opsahl-powergrid & \numprint{0.2} & \numprint{0.2} & \numprint{0.2} & \numprint{1.7} & \numprint{0.9} & \numprint{1.4}\\
samoa & \numprint{0.8} & \numprint{0.9} & \numprint{0.9} & \numprint{2.6} & \numprint{2.9} & \numprint{5.5}\\
comores & \numprint{0.5} & \numprint{0.5} & \numprint{0.6} & \numprint{1.1} & \numprint{2.3} & \numprint{8.6}\\
\bottomrule
\end{tabular}

\end{subtable}\hfill
\begin{subtable}[t]{.5\textwidth}
\centering
\captionabove{Undirected weighted}
\begin{tabular}{lrrrrrr}
\toprule
Graph & \multicolumn{3}{c}{Greedy-H} & \multicolumn{3}{c}{Greedy-LS-H}\\
\hfill $k$ & $5$ & $10$ & $50$ & $5$ & $10$ & $50$\\
\midrule
marshall-islands & \textless\numprint{0.1} & \textless\numprint{0.1} & \textless\numprint{0.1} & \textless\numprint{0.1} & \textless\numprint{0.1} & \textless\numprint{0.1}\\
micronesia & \textless\numprint{0.1} & \textless\numprint{0.1} & \textless\numprint{0.1} & \textless\numprint{0.1} & \textless\numprint{0.1} & \numprint{0.2}\\
kiribati & \textless\numprint{0.1} & \textless\numprint{0.1} & \textless\numprint{0.1} & \textless\numprint{0.1} & \textless\numprint{0.1} & \numprint{0.3}\\
opsahl-powergrid & \numprint{0.2} & \numprint{0.2} & \numprint{0.2} & \numprint{1.7} & \numprint{0.9} & \numprint{1.4}\\
samoa & \numprint{0.8} & \numprint{0.9} & \numprint{0.9} & \numprint{2.6} & \numprint{2.9} & \numprint{5.5}\\
comores & \numprint{0.5} & \numprint{0.5} & \numprint{0.6} & \numprint{1.1} & \numprint{2.3} & \numprint{8.6}\\
\bottomrule
\end{tabular}

\end{subtable}\medskip

\begin{subtable}[t]{.485\textwidth}
\centering
\caption{Directed unweighted}
\begin{tabular}{lrrrrrr}
\toprule
Graph & \multicolumn{3}{c}{Greedy-H} & \multicolumn{3}{c}{Greedy-LS-H}\\
\hfill $k$ & $5$ & $10$ & $50$ & $5$ & $10$ & $50$\\
\midrule
marshall-islands & \textless\numprint{0.1} & \textless\numprint{0.1} & \textless\numprint{0.1} & \textless\numprint{0.1} & \textless\numprint{0.1} & \textless\numprint{0.1}\\
micronesia & \textless\numprint{0.1} & \textless\numprint{0.1} & \textless\numprint{0.1} & \textless\numprint{0.1} & \textless\numprint{0.1} & \numprint{0.2}\\
kiribati & \textless\numprint{0.1} & \textless\numprint{0.1} & \textless\numprint{0.1} & \textless\numprint{0.1} & \textless\numprint{0.1} & \numprint{0.3}\\
samoa & \numprint{0.8} & \numprint{0.9} & \numprint{0.9} & \numprint{2.6} & \numprint{1.9} & \numprint{5.9}\\
comores & \numprint{0.5} & \numprint{0.5} & \numprint{0.6} & \numprint{1.1} & \numprint{3.3} & \numprint{10.5}\\
opsahl-openflights & \textless\numprint{0.1} & \textless\numprint{0.1} & \textless\numprint{0.1} & \textless\numprint{0.1} & \textless\numprint{0.1} & \numprint{0.2}\\
tntp-ChicagoRegional & \numprint{2.7} & \numprint{2.9} & \numprint{3.2} & \numprint{11.5} & \numprint{20.6} & \numprint{85.8}\\
\bottomrule
\end{tabular}

\end{subtable}\hfill
\begin{subtable}[t]{.515\textwidth}
\centering
\caption{Directed weighted}
\begin{tabular}{lrrrrrr}
\toprule
Graph & \multicolumn{3}{c}{Greedy-H} & \multicolumn{3}{c}{Greedy-LS-H}\\
\hfill $k$ & $5$ & $10$ & $50$ & $5$ & $10$ & $50$\\
\midrule
marshall-islands & \textless\numprint{0.1} & \textless\numprint{0.1} & \textless\numprint{0.1} & \textless\numprint{0.1} & \textless\numprint{0.1} & \textless\numprint{0.1}\\
micronesia & \textless\numprint{0.1} & \textless\numprint{0.1} & \textless\numprint{0.1} & \textless\numprint{0.1} & \textless\numprint{0.1} & \numprint{0.2}\\
kiribati & \textless\numprint{0.1} & \textless\numprint{0.1} & \textless\numprint{0.1} & \textless\numprint{0.1} & \textless\numprint{0.1} & \numprint{0.3}\\
samoa & \numprint{0.8} & \numprint{0.9} & \numprint{0.9} & \numprint{2.6} & \numprint{1.9} & \numprint{5.9}\\
comores & \numprint{0.5} & \numprint{0.5} & \numprint{0.6} & \numprint{1.1} & \numprint{3.3} & \numprint{10.5}\\
opsahl-openflights & \textless\numprint{0.1} & \textless\numprint{0.1} & \textless\numprint{0.1} & \textless\numprint{0.1} & \textless\numprint{0.1} & \numprint{0.2}\\
tntp-ChicagoRegional & \numprint{2.7} & \numprint{2.9} & \numprint{3.2} & \numprint{11.5} & \numprint{20.6} & \numprint{85.8}\\
\bottomrule
\end{tabular}

\end{subtable}
\end{table}

\section{Running Times -- Group-Closeness Maximization}
\label{sec:gh-gc:running-times-gc}

\begin{table}[H]
\centering\scriptsize
\captionabove{Running time (s) of \gslsc and \greedylsc
on the complex networks of \Cref{tab:gh-gc-apx:large-inst-gc-cplx}.}
\label{tab:gh-gc-apx:time-gc-cplx}

\begin{subtable}[t]{.5\textwidth}
\centering
\caption{Undirected unweighted}
\begin{tabular}{lrrrrrr}
\toprule
Graph & \multicolumn{3}{c}{GS-LS-C} & \multicolumn{3}{c}{Greedy-LS-C}\\
\hfill $k$ & $5$ & $10$ & $50$ & $5$ & $10$ & $50$\\
\midrule
loc-brightkite\_edges & \numprint{11.8} & \numprint{22.1} & \numprint{146.4} & \numprint{11.5} & \numprint{20.8} & \numprint{110.9}\\
douban & \numprint{35.0} & \numprint{59.4} & \numprint{222.0} & \numprint{26.3} & \numprint{43.5} & \numprint{202.5}\\
petster-cat-household & \numprint{32.7} & \numprint{66.1} & \numprint{363.2} & \numprint{32.2} & \numprint{63.2} & \numprint{341.5}\\
wikipedia\_link\_fy & \numprint{100.2} & \numprint{102.5} & \numprint{476.9} & \numprint{27.5} & \numprint{50.3} & \numprint{434.7}\\
wikipedia\_link\_ckb & \numprint{19.7} & \numprint{103.2} & \numprint{767.2} & \numprint{19.4} & \numprint{34.1} & \numprint{718.3}\\
livemocha & \numprint{58.1} & \numprint{86.3} & \numprint{713.0} & \numprint{46.5} & \numprint{58.2} & \numprint{604.9}\\
\bottomrule
\end{tabular}

\end{subtable}\hfill
\begin{subtable}[t]{.5\textwidth}
\centering
\caption{Directed unweighted}
\begin{tabular}{lrrrrrr}
\toprule
Graph & \multicolumn{3}{c}{GS-LS-C} & \multicolumn{3}{c}{Greedy-LS-C}\\
\hfill $k$ & $5$ & $10$ & $50$ & $5$ & $10$ & $50$\\
\midrule
wikipedia\_link\_mi & \textless\numprint{0.1} & \numprint{0.8} & \numprint{2.9} & \numprint{0.1} & \numprint{0.2} & \numprint{1.8}\\
foldoc & \numprint{2.3} & \numprint{3.5} & \numprint{0.5} & \numprint{1.7} & \numprint{2.2} & \numprint{5.7}\\
wikipedia\_link\_so & \numprint{0.7} & \numprint{1.5} & \numprint{23.7} & \numprint{0.5} & \numprint{0.9} & \numprint{3.3}\\
wikipedia\_link\_lo & \numprint{0.8} & \numprint{1.6} & \numprint{26.0} & \numprint{0.5} & \numprint{2.1} & \numprint{13.5}\\
wikipedia\_link\_co & \numprint{0.8} & \numprint{1.7} & \numprint{42.9} & \numprint{1.2} & \numprint{1.7} & \numprint{18.5}\\
soc-Epinions1 & \numprint{4.2} & \numprint{6.9} & \numprint{30.1} & \numprint{3.6} & \numprint{6.0} & \numprint{28.2}\\
slashdot-zoo & \numprint{4.1} & \numprint{6.4} & \numprint{19.9} & \numprint{3.4} & \numprint{7.1} & \numprint{15.4}\\
web-NotreDame & \numprint{14.9} & \numprint{37.3} & \numprint{1106.5} & \numprint{14.4} & \numprint{23.4} & \numprint{388.6}\\
wikipedia\_link\_jv & \numprint{22.9} & \numprint{97.1} & \numprint{30.0} & \numprint{17.7} & \numprint{14.5} & \numprint{49.9}\\
\bottomrule
\end{tabular}

\end{subtable}
\end{table}

\begin{table}[H]
\centering\scriptsize
\captionabove{Running time (s) of \gslsc and \greedylsc
on the high-diameter networks of \Cref{tab:gh-gc-apx:large-inst-gc-high-diam}.}
\label{tab:gh-gc-apx:time-gc-high-diam}

\begin{subtable}[t]{.5\textwidth}
\centering
\caption{Undirected unweighted}
\begin{tabular}{lrrrrrr}
\toprule
Graph & \multicolumn{3}{c}{GS-LS-C} & \multicolumn{3}{c}{Greedy-LS-C}\\
\hfill $k$ & $5$ & $10$ & $50$ & $5$ & $10$ & $50$\\
\midrule
opsahl-powergrid & \numprint{0.9} & \numprint{1.1} & \numprint{13.4} & \numprint{0.7} & \numprint{0.4} & \numprint{3.7}\\
andorra & \numprint{3.6} & \numprint{8.9} & \numprint{55.1} & \numprint{1.9} & \numprint{3.9} & \numprint{26.7}\\
seychelles & \numprint{1.6} & \numprint{5.3} & \numprint{26.7} & \numprint{0.9} & \numprint{3.4} & \numprint{25.0}\\
liechtenstein & \numprint{10.8} & \numprint{21.2} & \numprint{56.3} & \numprint{2.2} & \numprint{16.4} & \numprint{38.2}\\
comores & \numprint{1.2} & \numprint{5.0} & \numprint{22.9} & \numprint{1.4} & \numprint{4.9} & \numprint{18.0}\\
faroe-islands & \numprint{33.9} & \numprint{77.2} & \numprint{313.5} & \numprint{25.3} & \numprint{96.4} & \numprint{268.4}\\
\bottomrule
\end{tabular}

\end{subtable}\hfill
\begin{subtable}[t]{.5\textwidth}
\centering
\caption{Directed unweighted}
\begin{tabular}{lrrrrrr}
\toprule
Graph & \multicolumn{3}{c}{GS-LS-C} & \multicolumn{3}{c}{Greedy-LS-C}\\
\hfill $k$ & $5$ & $10$ & $50$ & $5$ & $10$ & $50$\\
\midrule
andorra & \numprint{5.2} & \numprint{6.2} & \numprint{5.0} & \numprint{4.2} & \numprint{3.5} & \numprint{26.0}\\
seychelles & \numprint{1.4} & \numprint{5.1} & \numprint{21.0} & \numprint{0.8} & \numprint{3.3} & \numprint{17.6}\\
liechtenstein & \numprint{17.7} & \numprint{14.4} & \numprint{59.7} & \numprint{4.0} & \numprint{15.5} & \numprint{41.3}\\
comores & \numprint{1.7} & \numprint{4.3} & \numprint{7.7} & \numprint{1.2} & \numprint{4.1} & \numprint{19.2}\\
faroe-islands & \numprint{20.2} & \numprint{77.0} & \numprint{254.1} & \numprint{25.7} & \numprint{44.8} & \numprint{189.4}\\
opsahl-openflights & \textless\numprint{0.1} & \numprint{0.1} & \numprint{1.0} & \textless\numprint{0.1} & \numprint{0.1} & \numprint{0.6}\\
tntp-ChicagoRegional & \numprint{45.3} & \numprint{151.4} & \numprint{0.3} & \numprint{32.5} & \numprint{68.2} & \numprint{304.3}\\
\bottomrule
\end{tabular}

\end{subtable}\medskip

\begin{subtable}[t]{.5\textwidth}
\centering
\caption{Undirected weighted}
\begin{tabular}{lrrrrrr}
\toprule
Graph & \multicolumn{3}{c}{GS-LS-C} & \multicolumn{3}{c}{Greedy-LS-C}\\
\hfill $k$ & $5$ & $10$ & $50$ & $5$ & $10$ & $50$\\
\midrule
andorra & \numprint{20.5} & \numprint{35.5} & \numprint{182.0} & \numprint{4.5} & \numprint{10.9} & \numprint{64.3}\\
seychelles & \numprint{2.6} & \numprint{13.7} & \numprint{93.1} & \numprint{2.3} & \numprint{3.3} & \numprint{62.6}\\
liechtenstein & \numprint{3.8} & \numprint{8.6} & \numprint{230.3} & \numprint{4.1} & \numprint{27.0} & \numprint{265.6}\\
DC & \numprint{7.8} & \numprint{18.9} & \numprint{473.2} & \numprint{9.9} & \numprint{14.0} & \numprint{98.3}\\
comores & \numprint{2.3} & \numprint{10.1} & \numprint{140.1} & \numprint{2.3} & \numprint{9.6} & \numprint{55.3}\\
faroe-islands & \numprint{17.1} & \numprint{137.5} & \numprint{907.0} & \numprint{15.6} & \numprint{27.4} & \numprint{411.3}\\
\bottomrule
\end{tabular}

\end{subtable}\hfill
\begin{subtable}[t]{.5\textwidth}
\centering
\caption{Directed weighted}
\begin{tabular}{lrrrrrr}
\toprule
Graph & \multicolumn{3}{c}{GS-LS-C} & \multicolumn{3}{c}{Greedy-LS-C}\\
\hfill $k$ & $5$ & $10$ & $50$ & $5$ & $10$ & $50$\\
\midrule
andorra & \numprint{5.9} & \numprint{16.0} & \numprint{129.3} & \numprint{6.4} & \numprint{5.8} & \numprint{52.8}\\
seychelles & \numprint{2.1} & \numprint{2.8} & \numprint{59.2} & \numprint{2.3} & \numprint{2.7} & \numprint{29.6}\\
liechtenstein & \numprint{3.7} & \numprint{16.9} & \numprint{227.5} & \numprint{3.8} & \numprint{22.2} & \numprint{167.9}\\
comores & \numprint{1.9} & \numprint{7.0} & \numprint{90.0} & \numprint{2.2} & \numprint{10.7} & \numprint{28.1}\\
faroe-islands & \numprint{16.2} & \numprint{148.1} & \numprint{696.2} & \numprint{15.3} & \numprint{27.2} & \numprint{98.5}\\
\bottomrule
\end{tabular}

\end{subtable}
\end{table}



