\chapter*{Introduction}
%
As we saw in \Cref{part:single-vertex-centrality},
finding highly central vertices in a graph is a fundamental
problem in network analysis (see also Ref.~\cite{newman2018networks}).
To this end, several
centrality measures have been introduced over the past decades (see
Ref.~\cite{DBLP:journals/im/BoldiV14} and \Cref{sec:centrality-measures}).
The problem of identifying the top-$k$ most central vertices in
a graph has been widely studied
\cite{DBLP:journals/tkdd/BergaminiBCMM19,
DBLP:conf/faw/OkamotoCL08,DBLP:conf/icde/OlsenLH14,DBLP:conf/www/LeeC14}.
However, many network analysis applications do not require the $k$
most \emph{individually} central vertices, but rather a \emph{group}
of $k$ vertices that is central \emph{as a whole}
\cite{DBLP:journals/toc/KempeKT15,
DBLP:journals/isci/ZhaoWLTG17,DBLP:journals/pe/GkantsidisMS06,DBLP:conf/icde/LiLCCDZ15,
yan2006efficient}.

Everett and Borgatti~\cite{everett1999centrality}
were the first to extend the centrality concept to groups of vertices and
defined group-degree, group-closeness, group-betweenness, and flow betweenness.
Group centrality maximization problems are often \np-hard: group-degree
maximization can be reduced from vertex cover~\cite{DBLP:conf/alenex/AngrimanGBZGM20},
for group-closeness see Ref.~\cite{DBLP:conf/adc/ChenWW16} and for group-harmonic
see Ref.~\cite{DBLP:conf/alenex/AngrimanBDGGM21}.
Consequently, exact methods for these problems such as ILP
solvers~\cite{DBLP:conf/alenex/BergaminiGM18} take too long on graphs with
non-trivial size -- a few thousands of vertices/edges or more.
Thus, group centrality maximization is generally approached via
approximation~\cite{DBLP:conf/kdd/MahmoodyTU16,DBLP:conf/www/0002PSYZ19}
or heuristics~\cite{DBLP:conf/alenex/BergaminiGM18,DBLP:conf/adc/ChenWW16}.
Yet, existing algorithms for group centrality maximization fail to scale to
large real-world networks. In addition, early attempts to attribute a constant-factor
approximation to a popular greedy algorithm for group-closeness maximization
were flawed (see \Cref{sec:group-harm-clos-max:gc-prel-disc}), leaving the open
questions of whether this problem can be approximated and, if so, how well.
Another weakness of existing group centrality measures is that, without proper
adjustments, they are not designed to handle disconnected graphs.

In the following, we address the lack of scalability of group centrality
maximization algorithms from two directions: (i) in the context of
group-closeness centrality, we introduce new fast local search heuristics to
compute highly central groups of vertices in large graphs
(\Cref{ch:group-closeness-local-search}) and (ii) we introduce a novel
group centrality measure called GED-Walk as well as efficient parallel
algorithms compute it and to approximate the group of vertices with highest
GED-Walk centrality (\Cref{ch:ged-walk}).
%
Concerning the group-closeness approximation issue, building on top of the
results presented in Ref.~\cite{DBLP:conf/alenex/AngrimanBDGGM21}, we provide
an efficient implementation of the first approximation algorithm for this
problem (\Cref{ch:group-harm-clos-max}). Finally, we address the lack of
electrical group centrality measures for disconnected graphs by extending
forest closeness to disconnected graphs (\Cref{ch:ged-walk}); as its
single-vertex counterpart, group forest closeness handles disconnected graphs
out of the box.

\paragraph{Contribution and Outline}
In \Cref{ch:group-closeness-local-search}, we introduce new fast local search
heuristics for group-closeness to compute highly central groups of vertices.
Our heuristics start from an initial a group of vertices $S$ and perform
exchanges between vertices within $S$ and the rest of the graph until they
reach a local optim\change{um}. Compared to the state-of-the-art greedy
algorithm~\cite{DBLP:conf/alenex/BergaminiGM18}, experimental results show
that, on unweighted graphs, our strategy is two orders of magnitude faster and
achieves $\numprint{99.4}\%$ of the solution quality; on weighted graphs, it
yields solutions of $\onedigit{\wGSQualInc}\%$ \emph{higher} quality while also
being $\onedigit{\gsPSevenFiveKSpeed}\times$ faster.
Furthermore, our algorithms handle graphs with hundreds of million of edges
in only around ten minutes, while the greedy algorithm requires more than
ten \emph{hours}.

Heuristics, however, do not provide any approximation guarantees, leaving
the question of approximability for group-closeness maximization unsettled.
In \Cref{ch:group-harm-clos-max}, we address this issue.
By exploiting the theoretical results
from Ref.~\cite{DBLP:conf/alenex/AngrimanBDGGM21}, we build the first
approximation algorithm for group-closeness maximization by adapting to
group-closeness the local search algorithm for $k$-Median by Arya
\etal~\cite{DBLP:journals/siamcomp/AryaGKMMP04}. In
addition, we also address \emph{group-harmonic} maximization -- to
the best of our knowledge, we are the first to study this optimization problem.
We implement an efficient greedy algorithm as well as a local search
approximation algorithm to maximize group-harmonic.
In our experimental study we show that, compared to a greedy strategy, our local
search approximation algorithm yields solutions with superior quality at the cost
of additional running time. In particular, local search is one to three orders
of magnitude slower than greedy, which is expected due to the approximation
guarantees. Indeed, local search approximation algorithms often cut greedy's
(empirical) gap to optimality by half or more.

Finally, in \Cref{ch:ged-walk}, we introduce two new group centrality
measures: GED-Walk, a novel group centrality
measure inspired by Katz centrality, and group forest closeness, that is, forest
closeness (see \Cref{ch:electrical-closeness}) extended to sets of vertices.
Similarly to Katz centrality, GED-Walk takes into account walks of any length
and gives to shorter walks greater importance than to longer walks. Since it is
not based on shortest paths, GED-Walk can be optimized significantly faster
(for groups with moderate size) compared to shortest-path based measures such
as group-closeness, group-harmonic or group-betweenness.
%
We describe efficient parallel algorithms to
compute the GED-Walk score of a given group and to efficiently approximate the
group of vertices with highest GED-Walk centrality. Experimental results show
that GED-Walk improves the precision of popular graph mining applications
such as semi-supervised vertex classification and graph
classification~\cite{DBLP:journals/tnn/ChapelleSZ09}. Also, our
algorithm for maximizing GED-Walk (in approximation) is two orders of magnitude
faster than state-of-the-art algorithms for group-betweenness
maximization~\cite{DBLP:conf/kdd/MahmoodyTU16} and, for group sizes up to 100
vertices, one to two orders of magnitude faster than group-harmonic and
group-closeness maximization~\cite{DBLP:conf/alenex/BergaminiGM18}.

Group forest closeness, in turn, is defined as the reciprocal of group
forest farness, \ie the sum of forest distances from a group $S$ to the
other vertices. Maximizing this measure turns out to be \np-hard, so we adapt
the greedy approximate algorithm from Li \etal~\cite{DBLP:conf/www/0002PSYZ19}
for group electrical closeness to group forest closeness.
Semi-supervised vertex classification results on disconnected graphs indicate
that, in comparison to existing measures, group forest closeness improves the
precision to a much greater extent.
