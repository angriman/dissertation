\pagestyle{plain}
\begin{center}
  \textsc{Abstract}
\end{center}
%
Network analysis is a collection of techniques aimed to unveil non-trivial
insights from networked data by studying relationship patterns between
the entities of a network.
Among the insights that can be extracted from a network, a popular one is to
quantify how important an entity is with respect to the others according to
some importance criteria -- \eg social networks often have
participants that are more influential than others or whose removal from the
network would imply a major disruption of the communication flow.
Another one is to find the most suitable matching partner for each participant
of a network knowing the pairwise preferences (or the compatibility) of the
participants to be matched with each other -- which can be formalized as the
well-known maximum weighted matching problem, or MWM. Think of matchmaking
platforms or pairing players in a chess tournament.

Because the notion of importance is tied to the application under consideration,
numerous \emph{centrality measures} have been introduced.
However, many of these measures were conceived in a time when computing power was very
limited and networks were much smaller compared to today's, and thus scalability
to large datasets was not considered. Today, massive networks with
millions of edges are ubiquitous and a complete exact computation for
traditional centrality measures -- still widely used to extract meaningful
information from large datasets -- often requires an excessive amount of time.
This issue is amplified if our objective is to find the \emph{group} of $k$
vertices that is the most central \emph{as a group}, which is useful to
applications such as optimizing warehouse locations or influence maximization.
Scalable algorithms to identify highly central (groups of) vertices
on massive graphs are thus of pivotal importance for large-scale network analysis.
%
In addition to their size, today's networks often evolve over time and
this poses the challenge of efficiently updating the findings we gathered on
the network after a change occurs.
Hence, efficient \emph{dynamic} algorithms
are essential for modern network analysis
pipelines.

In this work, we propose scalable algorithms for the two
aforementioned challenges, namely: identifying important vertices
in a network
and efficiently updating them in evolving networks.
In real-world graphs with up to hundreds of millions of edges, most of our
algorithms require seconds to a few minutes to perform these tasks, improving
significantly over the state of the art.

Concerning MWM, solving this problem optimally in large
graphs is computationally too expensive, and
fast inexact algorithms for MWM are abundant in the literature as well as dynamic
algorithms to maintain an (approximate) MWM.
\change{In the dynamic case}, however, little effort was devoted to actually
implementing these algorithms in practice, leaving their actual performance
unknown.
In this work, we extend a state-of-the-art approximation algorithm for MWM to
\change{dynamic graphs}; experiments show that our dynamic MWM algorithm handles
multiple edge updates in graphs with billion edges in only a fraction of a
second.

%The algorithms presented in this thesis were developed following
%the \emph{algorithm engineering} paradigm, meaning that, in addition to the
%theoretical analysis, implementation and systematic experimental evaluation
%are fundamental parts of the development process.
%
%This thesis is organized into five parts. While the first and the last parts
%serve as introduction and concluding remarks, respectively, the other three
%target different families of network analysis problems. Specifically,
%\Cref{part:single-vertex-centrality} concerns single-vertex centrality
%measures, \Cref{part:group-centrality} addresses group centrality maximization,
%and \Cref{part:matching} deals with approximate MWM in dynamic graphs.
%In the following, we provide a brief summary of the contents presented
%in \Crefrange{part:single-vertex-centrality}{part:matching}.
%
%\paragraph{Algorithms for Single-Vertex Centrality Measures}
%%
%Among the different types of centralities available in the literature, many of
%them consider the shortest-path distance between vertices. Closeness centrality
%is one of the oldest, yet still popular, centrality measures: it is defined as
%the reciprocal of the average (shortest-path) distance to all other nodes -- a
%vertex is central if it \emph{close} to the others.
%%
%Computing closeness for all vertices requires to solve All-Pairs
%Shortest Paths, which is notoriously unfeasible in large-scale instances.
%However, since applications often demand to identify the most central vertices
%rather than the score of each one, efficient algorithms to find the top-$k$
%vertices with highest closeness have been
%proposed~\cite{DBLP:journals/tkdd/BergaminiBCMM19}.
%In \Cref{ch:dyn-topk}, we consider the problem of maintaining the top-$k$
%closeness ranking in time-evolving graphs. Our dynamic algorithms exploit the
%information computed on an initial snapshot of the network to skip unnecessary
%work. Experimental results show that our approach is on average two to four
%orders of magnitude faster than a static recomputation.
%
%\Cref{ch:betweenness-approx} deals with parallel approximation schemes for
%another popular shortest-path based
%centrality measure: betweenness centrality. Betweenness assigns
%importance to a vertex depending on its participation in the shortest
%paths of the network. Because \emph{all} shortest paths are considered,
%this measure is quite expensive to compute, even for a single vertex.
%This motivated the development of approximation algorithms for
%betweenness and, among them, we consider the state-of-the-art \kadabra algorithm.
%\kadabra is an \emph{adaptive} sampling algorithm, \ie it keeps drawing
%samples -- in this case, shortest paths -- until a certain \emph{stopping
%condition} is met. Consequently, the required number of samples to achieve
%the desired approximation is not known in advance but depends on the
%samples drawn so far.
%%
%Trivial parallelization schemes incur to high synchronization overheads and
%do not scale to large numbers of threads. Hence, in this chapter, we implement
%novel epoch-based parallel adaptive sampling algorithms that aim to minimize
%the synchronization costs between threads.
%To demonstrate their effectiveness, we turn to \kadabra\ -- although
%our techniques can easily be adapted to other adaptive sampling algorithms.
%Experimental results show that, compared to a traditional OpenMP-based
%parallelization strategy, our algorithms achieve much higher parallel speedups.
%
%Shortest-path based measures are often criticized because of the assumption
%that whatever flows through the network follows shortest paths
%exclusively~\cite{DBLP:journals/socnet/Borgatti05}, which is not always
%the case.
%This led to the introduction of a whole new family of \emph{electrical}
%centrality measures based on \emph{resistance distance}. Resistance distance
%interprets the underlying graph as an electrical network and evaluates the
%distance between two vertices by taking into account \emph{all}
%paths between them.
%Unfortunately, computing electrical centrality measures exactly requires, in
%general, matrix inversion,  which is impractical for large instances.
%In \Cref{ch:electrical-closeness}, we present algorithms
%based on sampling of uniform spanning trees to approximate two popular
%electrical centrality measures: electrical closeness and forest closeness.
%Compared to state-of-the-art approaches, our algorithms not only are faster
%but also yield solutions with superior quality in terms of absolute error.
%
%
%\paragraph{Algorithms for Group Centrality Measures}
%%
%Applications such as optimizing warehouse locations or identifying
%the potentially best promoters for a product in a social network do not require
%to find the most \emph{individually} central vertices but rather the
%group of vertices that provides the widest \enquote{coverage} of the entire
%graph. To quantify the importance of a whole group of vertices in a network,
%group centrality measures were introduced.
%
%Finding the most central group of $k$ vertices (also known as the group centrality
%maximization problem) is, however, \np-hard for many popular group centrality measures.
%For this reason, proposed solutions resort to approximation or heuristics.
%Existing algorithms present limitations such as very limited scalability to large
%networks and, in some cases, lack of approximation guarantees.
%Furthermore, existing group centrality measures do not handle disconnected graphs.
%In this part, we address these issues.
%
%\Cref{ch:group-closeness-local-search,ch:group-harm-clos-max} are devoted to
%group-closeness. Existing maximization algorithms for this measure are heuristics
%that do not scale to large graphs -- the fastest one needs several
%hours for graphs with hundreds
%of millions of vertices. In \Cref{ch:group-closeness-local-search}, we
%introduce new local search heuristics that require minutes (but mostly a few
%seconds) to find highly-central groups of vertices even in bigger graphs.
%\Cref{ch:group-harm-clos-max}, in turn, introduces the first approximation
%algorithm for group-closeness. This algorithm is based on local search as well;
%however, its time complexity is at least quadratic in the number of vertices in
%the graph, which makes it suitable for networks with moderate size only.
%This chapter also addresses the lack of group centrality measures for disconnected
%graphs by extending harmonic centrality\footnote{Harmonic centrality is defined
%as the harmonic sum of the distances from one vertex to the others (see
%\Cref{eq:def:harmonic}) and it is suitable also for disconnected graphs.}
%to the group case. We develop approximation algorithms for group-harmonic
%maximization as well.
%
%Finally, \Cref{ch:ged-walk} addresses the limitations in scalability of
%existing group centrality maximization algorithms by proposing an alternative
%measure. Because shortest-path based centrality measures are inherently
%expensive to compute due to complexity-theoretic barriers, we devise GED-Walk,
%a new group centrality measure based on \emph{walks} inspired by Katz
%centrality (see \Cref{eq:def:katz}). In addition, we implement an efficient
%parallel
%approximation algorithm to maximize GED-Walk which, in our experiments, turns
%out to be much faster compared to state-of-the-art maximization algorithms for
%existing shortest-path based measures.
%In this chapter, we also introduce group forest closeness, the first
%\emph{electrical} group centrality measure that handles disconnected graphs out
%of the box, and we adapt the greedy algorithm by Li
%\etal~\cite{DBLP:conf/www/0002PSYZ19} to approximate the group forest
%maximization problem. Experiments on popular graph mining tasks show that
%GED-Walk and group forest closeness outperform existing measures in terms
%of precision in connected and disconnected graphs, respectively.
%
%\paragraph{Approximate Maximum Weighted Matching in Dynamic Networks}
%%
%MWM is one of the most popular combinatorial optimization problems with numerous
%practical applications. Even though computing optimal matchings \wrt
%cardinality or weight can be done in polynomial time, this is not feasible for
%large-scale networks. Thus, a wide range of approximation algorithms trading
%solution quality for a faster running time have been proposed.
%In the context of time-evolving graphs, various fully-dynamic algorithms that efficiently
%maintain an approximate MWM after a graph update have been introduced as well.
%However, little effort has been invested into implementing these algorithms in
%practice and no batch-dynamic algorithm for (approximate) MWM has been
%developed.
%
%\Cref{ch:dyn-mwm} focuses on the problem of maintaining a $\change{(1/2)}$-approximate MWM
%in fully-dynamic dynamic graphs. Limitations of existing approaches for this problem
%are: (i) high constant factors in their time complexity, (ii) the fact that none of
%them supports batch updates, and (iii) the lack of a practical implementation -- meaning
%that their actual performance on real-world networks has not been investigated.
%%
%We describe and implement a new batch-dynamic $\change{(1/2)}$-approximation algorithm for
%MWM based on the state-of-the-art \suitor
%algorithm~\cite{DBLP:conf/ipps/ManneH14} and its local edge domination
%strategy. We provide a detailed analysis of our algorithm and prove
%its approximation guarantee. Despite having a worst-case time complexity for a single
%edge update that is linear in the graph size (\ie equivalent to \suitor), our extensive
%experimental evaluation shows that our dynamic algorithm is much faster in practice.
%More precisely, for batches with up to $10^4$ edge updates, the speedup
%against a static recomputation is $10^2\times$ to $10^6\times$.
