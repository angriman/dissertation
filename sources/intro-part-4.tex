\chapter*{Introduction}
%
Stable Marriage is the popular problem introduced by Gale and
Shapley~\cite{DBLP:journals/tamm/GaleS13} of matching an equal number of women
and men, each of whom has ranked the participants of the opposite sex in order
of preference so that \change{no couple prefer each other to the partners they
are matched with}.
In other scenarios, however, there may not be a separation of the participants
into two classes (\eg same-gender couples or pairing players in a chess
tournament~\cite{kujansuu1999stable}), hence the Stable Roommates Problem
(SRP). SRP, originally described in the paper of Gale and
Shapley~\cite{DBLP:journals/tamm/GaleS13}, is essentially the stable marriage
problem with just one set and appears in a variety of practical
applications~\cite{iwama2007stable}. Every person in the set (of even cardinality $n$)
ranks the others in order of preference. The objective is to partition
the set into $n/2$ pairs of \enquote{roommates} such that no two
persons that are not roommates both prefer each other to their
current roommates -- \ie finding a stable \emph{matching}.

Some commercial and scientific applications, however, demand to find matchings
that are not only stable, but that also have the maximum weight, which leads to
the Maximum Weighted Matching problem
(MWM)~\cite{bisseling2020parallel,DBLP:conf/ecir/LaitangPB13,
DBLP:journals/ijprai/ConteFSV04,
DBLP:journals/pc/MonienPD00,wang2004bipartite,DBLP:conf/ismb/BorgwardtOSVSK05,
DBLP:journals/dc/HalldorssonKPR18,DBLP:journals/jpdc/Patt-ShamirRS12}.
To this end, several algorithms to solve MWM have been proposed;
because optimal solutions are expensive to compute,\footnote{The fastest
know algorithm for MWM on general graphs takes $\Oh(mn\log n)$
time~\cite{galil1986efficient}, other faster strategies exploit assumptions
on the input graph~\cite{DBLP:journals/talg/DuanPS18}.}
approximation algorithms with nearly-linear time complexity are
numerous in the literature~\cite{DBLP:conf/stacs/Preis99,DBLP:conf/ppam/ManneB07,
DBLP:conf/europar/BirnOSSS13,DBLP:conf/wea/DrakeH03,
DBLP:journals/ipl/DrakeH03,DBLP:conf/wea/MaueS07,DBLP:conf/ipps/ManneH14}.
As we argued in \Cref{sec:intro:motivation}, networks often change
over time, leading to the problem of updating an (approximate) MWM
after those change(s). The naive solution is to re-run a static algorithm
on the new graph; this, however, is a rather inefficient strategy:
as we saw in \Cref{ch:dyn-topk}, reusing the information computed on an
initial snapshot of the graph can reduce enormously the amount of work
required to update the results on the new graph.
This would be beneficial especially \change{for} applications dealing with
rapidly-evolving networks where multiple updates happen every second --
running a linear-time algorithm from scratch after each update would not
be practical.

Numerous dynamic algorithms for approximate MWM were
proposed~\cite{DBLP:conf/fsttcs/AnandBGS12,DBLP:conf/focs/GuptaP13,
DBLP:conf/innovations/StubbsW17,DBLP:conf/approx/CrouchS14} but
just a few of them have been
implemented~\cite{conf/acda/AngrimanMSU21} and none
supports edge updates in batches.
In \Cref{ch:dyn-mwm}, we address the problem of updating a $\change{(1/2)}$-approximate MWM
after multiple edge updates by presenting a new batch-dynamic algorithm
inspired by the \suitor algorithm by Manne and
Halappanavar~\cite{DBLP:conf/ipps/ManneH14}.
Despite having a worst-case time complexity that is linear in the size of the
input graph, our extensive experimental analysis suggests that, compared to the
state of the art~\cite{conf/acda/AngrimanMSU21},
our dynamic \suitor algorithm \change{traverses less edges} to handle
single edge updates and yields solutions with \quaRWLowest quality or higher.
Further, for batches with up to $10^4$ edge updates, it requires milliseconds
-- or less, for the batches of smaller size -- that is $10^2\times$ to $10^6\times$
faster than re-running \suitor from scratch.
